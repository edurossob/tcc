% \chapter{Exemplo de anexo}
% %=====================================================

% Os apêndices são uma extensão do texto, destacados deste para evitar descontinuidade na sequência lógica ou alongamento excessivo de determinado assunto ou tópico secundário dentro dos capítulos da dissertação ou da tese. São contribuições que servem para esclarecer, complementar, provar ou confirmar as ideias apresentadas no texto dos capítulos e que são importantes para a compreensão dos mesmos.

% Todos os apêndices devem vir após as referências bibliográficas e devem ser enumerados por letras maiúsculas (A, B, C, ...).

% %=====================================================

% \section{Uma Seção}

% \lipsum[20-23]

% %=====================================================

% \subsection{Uma Subseção}

% \lipsum[30-33]

% %=====================================================

% \subsubsection{Uma Subsubseção}

% \lipsum[30-33]

% %=====================================================
