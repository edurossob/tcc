\chapter{Proposal}
\label{cap:Proposal}


%=====================================================

This chapter shows the specific configurations utilized to train the proposed new skills
\section{Skills}
All of the following skills have their 

\subsection{Jump}

State space:
- Observasion size of 70 Float numbers (32 bits)
- Composed of all joint positions + torso height
- Stage of the underlying Step behavior

Action Space:
- Size 11
- Composed of all 

Reward:
- MAX (Displacement in the y-axis, 0)
'''

\subsection{Long Jump}

The Long Jump skill is inspired by the homonym sport, where the athlete must gain velocity in a short run and leap forward as far as possible from a takeoff point, called indicator board.

\begin{figure}[!htb]
\centering
\includegraphics[width=12cm]{long_jump.png}
\caption{Long Jump schematics}
\label{fig:long_jump_schematics}
\end{figure}

To train the agent within 3DSSL it is set an invisible line in a distance of 1/30th of the field length from the agent as the takeoff line. The line can be vizualied as one grass strip lenght, as showed in the image below.


\begin{figure}[!htb]
\centering
\includegraphics[width=12cm]{agent.png}
\caption{Agent on the starting point of a long jump}
\label{fig:3DSSL_robot}
\end{figure}

the reward is different before and after the 


State space:
- Composed of all joint positions + torso height
- Stage of the underlying Step behavior

Reward:
- Positive displacement in the y-axis, or 0
'''